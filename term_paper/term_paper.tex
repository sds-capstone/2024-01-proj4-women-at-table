% Options for packages loaded elsewhere
\PassOptionsToPackage{unicode}{hyperref}
\PassOptionsToPackage{hyphens}{url}
% !TeX program = pdfLaTeX
\documentclass[12pt]{article}
\usepackage{amsmath}
\usepackage{graphicx,psfrag,epsf}
\usepackage{enumerate}
\usepackage[]{natbib}
\usepackage{textcomp}


%\pdfminorversion=4
% NOTE: To produce blinded version, replace "0" with "1" below.
\newcommand{\blind}{0}

% DON'T change margins - should be 1 inch all around.
\addtolength{\oddsidemargin}{-.5in}%
\addtolength{\evensidemargin}{-1in}%
\addtolength{\textwidth}{1in}%
\addtolength{\textheight}{1.7in}%
\addtolength{\topmargin}{-1in}%

%% load any required packages here



% tightlist command for lists without linebreak
\providecommand{\tightlist}{%
  \setlength{\itemsep}{0pt}\setlength{\parskip}{0pt}}




\IfFileExists{bookmark.sty}{\usepackage{bookmark}}{\usepackage{hyperref}}
\IfFileExists{xurl.sty}{\usepackage{xurl}}{} % add URL line breaks if available
\hypersetup{
  pdftitle={Title here},
  pdfkeywords={, that do not appear in the title},
  hidelinks,
  pdfcreator={LaTeX via pandoc}}



\begin{document}


\def\spacingset#1{\renewcommand{\baselinestretch}%
{#1}\small\normalsize} \spacingset{1}


%%%%%%%%%%%%%%%%%%%%%%%%%%%%%%%%%%%%%%%%%%%%%%%%%%%%%%%%%%%%%%%%%%%%%%%%%%%%%%

\if0\blind
{
  \title{\bf Title here}

  \author{
        Author 1 \thanks{The authors gratefully acknowledge \ldots{}} \\
    Department of YYY, University of XXX\\
     and \\     Author 2 \\
    Department of ZZZ, University of WWW\\
      }
  \maketitle
} \fi

\if1\blind
{
  \bigskip
  \bigskip
  \bigskip
  \begin{center}
    {\LARGE\bf Title here}
  \end{center}
  \medskip
} \fi

\bigskip
\begin{abstract}
Financial inclusion is paramount for economic stability and resilience,
particularly in diverse regions like Sub-Saharan Africa, spanning low to
high-income countries and encompassing both resource-intensive and
non-resource-intensive economies. This study focuses on a crucial aspect
of financial resilience: the accessibility of emergency funds, defined
as having access to 1/20 of Gross National Income (GNI) per capita in
local currency within 30 days. Leveraging previous colleagues'
exploratory work on the Global Financial Inclusion Database 2021, our
objective is to mitigate inherent gender biases in the dataset by
rebalancing it with synthetic data, thereby enhancing fairness in
predicting emergency fund accessibility. Through predictive machine
learning modeling, we aim to contribute to the economic empowerment of
individuals in Sub-Saharan Africa, ultimately fostering resilience and
reducing disparities in access to essential financial resources.
\end{abstract}

\noindent%
{\it Keywords:} , that do not appear in the title

\vfill

\newpage
\spacingset{1.9} % DON'T change the spacing!

\hypertarget{introduction}{%
\section{Introduction}\label{introduction}}

In the realm of financial inclusion, the accessibility of emergency
funds plays a pivotal role in determining an individual's financial
stability and resilience, especially in developing countries. In this
project, our goal is to predict the possibility for people in
Sub-Saharan African countries to come up with emergency funds, defined
as 1/20 of GNI per capita in local currency, within a 30-day period.
This prediction serves as a crucial factor for establishing future
public financial policies, such as determining the eligibility of
individuals for loans and financial assistance. The significance of this
problem lies within its direct impact on the economic well-being and
empowerment of individuals in developing regions. According to the
Global Financial Inclusion (Global Findex) Database 2021 published by
the World Bank, only a little over half of people over 15 years of age
in developing economies could access extra funds within 30 days if faced
with unexpected expenses. Therefore, there is a pressing need to
understand the factors influencing this accessibility and eliminate
inherent bias in the dataset. By delving into this issue, we not only
contribute to enhancing financial inclusion but also aid in mitigating
the different effects of financial shocks on vulnerable populations.

Defining fairness is essential since the concept itself is relative
among different people. In the data science discourse, fairness
encompasses three key aspects: individual fairness, group fairness, and
causal fairness. Individual fairness focuses on preventing
discrimination against individuals with similar relevant
characteristics. This means ensuring that individuals in similar
situations receive similar outcomes from the model, regardless of
irrelevant factors. Group fairness aims to prevent disparities in
outcomes for different groups. This ensures equal opportunities for all
groups, regardless of their membership. Causal fairness goes beyond
simply observing disparities and delves into understanding their
underlying causes. It seeks to mitigate these root causes to achieve
fair outcomes within and across groups.

Our previous colleagues conducted analysis using the \textless AI \&
Equality\textgreater{} Human Rights Toolbox and used a Decision Tree
Classifier machine learning model implemented via Python to predict
access to emergency funds with 68\% accuracy. Their work laid a solid
foundation by exploring demographic and financial variables within the
dataset, and assessed the fairness of the decision tree classifier,
particularly concerning gender bias, and then applied various processing
techniques to enhance the fairness of the model.

Based on their work, our group aims to incorporate synthetic data to
rebalance the dataset, ensuring equitable representations amongst both
genders in the dataset. Our approach aims to consider a broader
selection of machine learning algorithms and mitigate the disparities
the previous colleagues found in the decision tree's classifier's
predictions and enhance fairness with synthetic data, ultimately better
predict access to emergency funds in south-Saharan Africa countries.

\hypertarget{background-on-sub-saharan-region}{%
\section{Background on Sub-Saharan
Region}\label{background-on-sub-saharan-region}}

Sub-Saharan Africa is a region characterized by a diverse economic
landscape, encompassing low, lower-middle, and upper-middle-income
countries. Demographically, Sub-Saharan Africa is marked by a rich
tapestry of cultures and a population exceeding 1.2 billion people. This
expansive and diverse demographic landscape includes 22 countries
grappling with fragility or conflict, posing unique challenges to
development efforts. Additionally, 13 small states within the region are
characterized by limited human capital, modest populations, and
constrained land areas.

According to the World Bank's definition, middle-income countries had a
per capita gross national income of more than US\$995.00 in the years
2015--17. Among the 35 countries included in our dataset, 20 are
classified as low-income countries, and 15 are classified as
middle-income countries. Additionally, 11 are classified as countries in
Fragile and Conflict-Affected Situations, which, by definition, have
experienced a peacekeeping or peace-building mission within the last
three years.

Sub-Saharan African countries not only differ in terms of economic
prosperity but also in economic structure and resource intensity.
Resource-intensive countries include both oil-exporting nations, where
net oil exports make up 30 percent or more of total exports, and
commodity exporters, where nonrenewable natural resources represent 25
percent or more of total exports. The divergence between
resource-intensive and non-resource-intensive countries became more
entrenched following the commodity price shock of 2015.
Non-resource-intensive countries have proven more resilient, supported
by their more diversified economies. On the other hand,
resource-intensive economies generally have a less diversified
structure, making them more susceptible to external shocks. In our
dataset, 6 of the countries are oil exporters, and 11 export other
commodities such as iron ore, copper, cotton, coffee, and sugar. The
remaining 18 countries are non-resource-intensive and their economies
are not reliant on exports.

In recent years, the Sub-Saharan Africa region has grappled with
significant economic challenges, including soaring inflation, pronounced
exchange rate pressures, debt vulnerabilities, and widening economic
disparities within the region. Therefore, addressing these structural
and economic disparities is imperative for tackling developmental issues
in the region.

\section{Verifications}
\label{sec:verify}

This section will be just long enough to illustrate what a full page of
text looks like, for margins and spacing.

\citet{Campbell02} \citeauthor{Schubert13}
\citetext{\citeyear{Schubert13}; \citealp{Chi81}}

\bibliographystyle{plain}
\bibliography{bibliography.bib}



\end{document}
